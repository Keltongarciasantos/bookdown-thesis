\begin{titlepage}
\begin{center}
\LARGE{\textbf{Universidade de Cabo Verde}}\\
\normalsize{Faculdade de Ciências e Tecnologias}\\[0.3cm]

\begin{figure}[h!]
    \centering
    \includegraphics[width=.3\linewidth]{img/hpi_logo.jpg}
\end{figure}
\vspace{4cm}

\LARGE{Mestrado em Matemática e Aplicações - Variante Finanças}\\[0.7cm]
\Huge{Análise e Previsão de Séries Temporais:\\ aplicações com dados da economia Caboverdiana.}

\vspace{3cm} 

\Large{\textbf{Kelton Santos}} \\[3pt]  
\vspace{0.5cm}
\large{\today} \\
Praia

\vspace{1cm}

\large{\textbf{Orientadora}}\\
M. Cristina Miranda\\
\vspace{0.5cm}
%\textbf{Advisor}\\
%\advisor\\
\end{center}
\end{titlepage}


\thispagestyle{empty}


\begin{abstract}
\justify 
\emph{Uma grande parte dos dados económico-financeiros podem ser representados em forma de séries temporais. Essas séries, que são caracterizadas por pontos observados com igual espaçamento temporal podem ser analisadas e utilizadas para fazer previsões sobre valores futuros. Existem vários métodos usados na análise e previsão de séries temporais, sendo que a maioria usa valores passados da série para criar um modelo usado para representação e previsão da série. Nesse trabalho usou-se o modelo ARIMA(1, 2, 1) para a análise e previsão do Produto Interno Bruto (PIB) de Cabo Verde de 1980 a 2020. O modelo utilizado produziu resultados muito próximos dos valores reais, e somente não conseguiu prever com eficiência a queda que aconteceu em 2020 em consequência da pandemia da COVID-19, visto que a  previsão de outliers é um desafio muito grande para qualquer método de previsão. Entretanto, novas técnicas computacionais (Machine Learning models) têm sido desenvolvidas para fazer face a essa dificuldade.}
\end{abstract}

% Alternative location to write Preface/Acknowledgements
% \newpage
% 
% \chapter*{Preface}
% 
% Thanks goes to.

\setcounter{page}{1}
